
\chapter{Conclusions}
\label{ch:conclusion}
The work presented in this thesis develops a model for the ensemble
zero locations and introduces two new ABF algorithms. The model for
the ensemble zeros expresses the zero locations as a function of the
INR. The model highlights the trade off between the interferer
suppression and white noise gain to produce minimum output power using
the ensemble MVDR beamformer. Numerical evaluation show that the model
predictions are a good match to the ensemble zero location, for a
range of INR values.

The UC MVDR ABF introduced in Ch.~\ref{ch:ucmvdr} improves interferer
and white noise suppression of the SMI MVDR ABF. The UC MVDR algorithm
moves the sample zeros radially to the unit circle satisfying the unit
circle constraint on the ensemble zeros. The UC MVDR approach differs
from the diagonal loading approach which is shown to move the sample
zeros along specific trajectories towards the CBF zero
locations. Simulation results show that the UC MVDR ABF yields deeper
notch at the interferer direction and improves WNG performance
compared to the SMI MVDR ABF, and the DL MVDR ABF with comparable DL
factor. Additionally, implementation of the UC MVDR in a snapshot deficient case and the unit circle DMR ABF are also discussed.

The DZ MVDR ABF introduced in Ch.~\ref{ch:dzmvdr} is a new approach to
notch broadening. The proposed algorithm exploits the property of ABF
beampattern near a second-order zero combined with subarray processing
to produce broad notches at interferer directions. Numerical
simulations evaluated the performance of the DZ MVDR ABF in presence
of stationary and moving interferers. The eigenanalysis of the sinc
taper matrix reveled that the CMT MVDR uses three DoFs per interferer
when the notch width is set equal to one resolution
width. Consequently, in presence of multiple interferers, the CMT MVDR
failed to suppress output power compared to the DZ MVDR ABF which uses
only two DoF per interferer.

%%% Local Variables:
%%% mode: latex
%%% TeX-master: "main"
%%% End:
