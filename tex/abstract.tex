
\newgeometry{top=0in}
\onehalfspacing
\chapter*{Abstract}
Analysis and Design of Adaptive Beamformers using Array Polynomial
\begin{flushleft}
by Saurav R. Tuladhar \\
\end{flushleft}

Adaptive beamformers (ABFs) place deep beampattern notches near
interferers to suppress the interferers’ power in the ABF output. The
sample matrix inversion (SMI) Minimum Variance Distortionless Response
(MVDR) ABF computes the beamformer weights by substituting the sample
covariance matrix (SCM) for the unknown ensemble covariance matrix in
the MVDR expression. Errors in the SCM estimate of interferer
direction due to limited sample support or interferer motion distorts
the ABF beampattern and degrades the ability of the ABF to suppress
the interferer. For beamformers using uniform linear arrays (ULA), the
beampattern can be represented as a polynomial. The array polynomial
is the $z$-transform of the beamformer weights. Evaluating the array
polynomial on the unit circle in the complex plane yields the
beampattern. For the ensemble MVDR beamforming using a ULA, the array
polynomial zeros are constrained to fall on the unit circle. But the
SMI MVDR polynomial zeros generally do not fall on the unit
circle. The first part of the dissertation develops a model for the
ensemble MVDR polynomial zero locations assuming a single interferer
present in white background noise. The model illuminates the trade off
balancing the interferer suppression and the white noise gain by the
ensemble MVDR. Secondly, the dissertation proposes the unit circle
MVDR (UC MVDR) beamformer which projects the SMI MVDR polynomial zeros
radially on to the unit circle to satisfy the constraint on the zeros
of ensemble MVDR polynomial. Numerical simulations show that the UC
MVDR beamformer suppresses interferers better than the SMI MVDR and
diagonal loaded MVDR beamformer and also improves the white noise gain
(WNG). Finally, the dissertation proposes the double zero (DZ) MVDR
ABF as a new approach to notch broadening. The array polynomial for
the DZ MVDR ABF has second-order zeros, producing broader and deeper
notches in the interferer direction. The DZ MVDR ABF outperforms the
SMI MVDR and covariance matrix tapered ABFs in simulations with
stationary and moving interferers.

\restoregeometry

%%% Local Variables:
%%% mode: latex
%%% TeX-master: "main"
%%% End:
