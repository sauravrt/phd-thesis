\chapter{Introduction}
\label{cha:introduction}
% Broad perspective
Passive sensor arrays sample propagating signals at spatially discrete
locations \cite{vtree2002oap,johnson1992array}. Signal processing
techniques spatially filter the observed data from the sensor
array to focus on propagating signals from a specified direction of
interest. Beamforming is the spatial filtering technique which passes
propagating signals arriving from the direction of interest while
rejecting noise and interfering signals. Beamforming on propagating
signals is analogous to bandpass filtering of temporal signals using
LTI filters. This dissertation focuses on beamforming for narrowband
planewave signals sampled using uniform linear arrays (ULA).

% Beamforming and ABF
Beamformers often operate in scenarios where the signal of interest
exists in the presence of loud interferers and noise, a common
scenario in passive sonar applications \cite{baggeroer1999passive,
  song2003null}. Conventional beamformers (CBFs) have limited ability
to suppress loud interferers which can leak through the high sidelobes
of the CBF beampattern and mask a weaker signal of interest. Adaptive
beamformers (ABF) can adjust their response according to the operating
scenario and pass signals of interest while attenuating interferers
and noise. ABFs use the knowledge of the data ensemble covariance
matrix (ECM) to place deep notches in the interferer direction to
suppress interferer power at output \cite{vtree2002oap}. In practice
the ECM is usually unknown and it is estimated from the observed
snapshot data by computing the sample covariance matrix
(SCM). Practical ABFs are implemented by substituting the SCM for the
ECM to compute the beamformer weights. The minimum variance distortionless
(MVDR) beamformer is a commonly used ABF \cite{capon1969mvdr}. The
practical MVDR beamformer implemented using the SCM is referred to as
the sample matrix inversion (SMI) MVDR ABF \cite{vtree2002oap}. This
dissertation focuses on the MVDR ABF for application in the passive sonar
environment.

% Issues in ABF
% may need to get in Mestre's work here

The SCM based ABFs rely on averaging a large number of snapshot data
to obtain a good estimate of the ECM using the SCM \cite{vtree2002oap,
  reed1974rapid}. Reed et al.\ show that the SMI MVDR ABF requires two
snapshots per sensor to keep the expected output SINR loss within $3$
dB of the ensemble output SINR \cite{reed1974rapid}. In practice,
either source motion or physical constraint due to the use of a large
array apertures limits the number of snapshots available to compute
the SCM. In passive sonar applications, it is common for the available
number of snapshots to be comparable to or even less than the number
of sensors \cite{baggeroer1999passive}. In such cases, inverting the
SCM is either unstable or even impossible. An SMI MVDR ABF implemented
with a limited number of snapshots suffers from beampattern
distortion, producing high side-lobes and a wider main-lobe
\cite{richmond96pdfarray,Carlson1988scm}. This beampattern distortion
results in a loss of capability to attenuate interferers and noise
power. Furthermore, interferer motion leads to mismatch between the
ABF notch direction and the true interferer direction resulting in
degraded interferer suppression \cite{baggeroer1999passive,
  cox2002adaptive, riba1997comm}.

Diagonal loading the SCM is one approach to address the issue of
limited or insufficient snapshots. Diagonal loading ensures that the
SCM is full rank and invertible to compute the MVDR beamformer
weights. In addition to enabling implementation of the SMI ABF in a
snapshot deficient scenario, diagonal loading also improves ABF
robustness against parameter mismatch \cite{vtree2002oap,
  mestre2005diagonal}. However, choosing an appropriate DL factor
requires knowledge of the interferer and noise power levels which are
unknown in practice. Several approaches have been proposed for
choosing the appropriate diagonal loading factor. Mestre and Lagunas
introduced a random matrix theory based estimator for the optimal DL
factor with focus on snapshot limited scenarios. However the procedure for finding the optimal DL factor has significantly higher computational complexity 
\cite{mestre2006finite}.

Notch broadening is a robust beamforming technique that addresses the
interferer direction mismatch issue by producing wide notch in the
beampattern at the interferer direction
\cite[Sec.~6.7.6]{vtree2002oap}. One approach to notch broadening
developed by Mailloux \cite{mailloux1995null} and Zatman
\cite{zatman1995null} augments the SCM such that the resulting ABF
produces broad notches. The covariance matrix taper (CMT) technique by
Guerci generalizes the SCM augmentation approach for notch broadening
\cite{guerci1999cmt}. Another notch broadening approach imposes
derivative constraints on the beampattern at the interferer direction
\cite[Sec.~6.7.1.4]{vtree2002oap}. The derivative constraint forces
the beampattern in the interferer direction to be flatter and hence
produce wider notch. Gershman et al.\ used the derivative constraint
approach to produce notch broadening using MVDR ABF
\cite{gershman1991synthesis}.

Designing a beamformer requires choosing the weights of each sensor to
obtain a beampattern with desirable properties. For ULAs the
beampattern can be represented in terms of a polynomial. The
$z$-transform of the weights for planewave beamformers using a ULA is
called the beamformer array polynomial \cite{Schelkunoff1943array,
  vtree2002oap, Steinberg1976}. This is analogous to the system
function representation of a discrete time (DT) LTI filter by taking
z-transform of the impulse response. As with DT LTI filters,
beamformers also have a pole-zero representation in the complex
plane. Continuing the analogy, the beampattern is obtained by
evaluating the array polynomial on the unit circle. The array
polynomial zeros produce notches in the beampattern and the zeros that
fall on the unit circle produce nulls in the beampattern.

The array polynomial for beamformers has seen limited
usage in array processing applications. Steinberg discusses the
polynomial representation of radiation pattern for uniform antenna
arrays and presents an approach to synthesize radiation patterns by
manipulating zero locations on the unit circle
\cite{Steinberg1976}. However, the methods proposed by Steinberg are
limited to ensemble implementation \cite{Steinberg1976}. The
Root-MUSIC algorithm for direction-of-arrival (DOA) estimation
exploits a polynomial representation of the MUSIC power spectrum
\cite{hwang2008root}. The zero locations of the power spectrum
polynomial on the unit circle are used to estimate the source
DOA. Several proposed DT adaptive notch filters constrain the filter
poles and zeros to render 'sharper' notches in their frequency
response \cite{Nehorai1985, Friedlander1984notch, Shynk1986}. However
these approaches are based on DT IIR filters while beamformers with
ULAs are analogous to DT FIR filters. This dissertation exploits the
array polynomial representation of MVDR ABF to develop new beamforming
algorithms.

% The zeros of the ensemble MVDR beamformer array polynomial are
% constrained to fall on the unit circle
% \cite{steinhardt2004stap}. However, the sample zeros of the SMI MVDR
% beamformer array polynomial generally do not fall on the unit
% circle. This paper develops a new beamformer by radially projecting
% the SMI MVDR zeros to the unit circle and therefore satisfying the
% constraint on the ensemble MVDR zeros. The proposed unit circle MVDR
% (UC MVDR) beamformer suppresses interferers better than the SMI MVDR
% and the diagonal loaded (DL) MVDR beamformer and at the same time
% improves white noise gain (WNG) performance.

% The polynomial representation approach from
% \cite{Schelkunoff1943array} hseen limited use in array processing
% applications. Root-MUSIC algorithm for direction-of-arrival (DoA)
% estimation used polynomial representation of the MUSIC power spectrum
% \cite{hwang2008root}. The zero locations of the power spectrum
% polynomial on the unit circle are used to estimate the source
% DOAs. However, unlike UC-MVDR, the polynomial zeros are not modified
% in Root-MUSIC. Several DT adaptive notch filter designs have been
% proposed which constrain the pole zero structure to improve filter
% response. Friedlander presented an IIR filter structure which forces
% the zeros to be on the unit circle to achieve deeper notches
% \cite{Friedlander1984notch}. Nehorai presents a pole zero constrained
% IIR notch filter which can produce sharper notches \cite{Nehorai1985}.


\section{Dissertation contribution}
\label{sec:contributions}
The contributions of this dissertation are as follows:
\begin{itemize}
\item Investigates the array polynomial zero locations for the MVDR
  beamformer. The work develops a model for the ensemble MVDR
  polynomial zero locations assuming a single interferer present in
  white background noise. The model illuminates the trade off balancing
  the interferer suppression and the white noise gain by the ensemble
  MVDR to produce the minimum output power.

\item Develops an MVDR ABF algorithm to produce a deeper notch in the
  beampattern at the interferer direction and simultaneously improve
  the white noise gain compared to the SMI MVDR ABF. The proposed
  algorithm forces the SMI MVDR ABF polynomial zeros to satisfy the
  unit circle constraint on the ensemble MVDR polynomial zeros. The
  unit circle MVDR beamformer is an improvement over diagonal loading
  and dominant mode rejection in that the new algorithm does not
  require an explicit choice of a design parameter.

\item Develops a new approach for notch broadening. The proposed
  algorithm exploits the properties of the MVDR beampattern near a
  second order zero combined with subarray processing. The algorithm
  implicitly identifies the interferer direction and imposes
  derivative constraint to broad notches by creating second-order
  zeros. The second-order zeros are generated without explicitly
  solving for the roots of array polynomial. 

 %  and imposes derivative constraint to to broaden notch. Subarray
 %  processing
\end{itemize}

\section{Dissertation outline}
\label{sec:outline}
The rest of the dissertation is organized as follows:
Ch.~\ref{ch:lit-rev} reviews essential background on beamforming using
sensor arrays, conventional and the adaptive beamformers, and
discusses several performance metrics. The final portion of the
chapter introduces the array polynomial representation for planewave
beamformers using a ULA. Ch.~\ref{ch:mvdr-polyn-zeros} discusses the
array polynomial representation of the ensemble MVDR ABF, the
properties of the ensemble zeros, and the SMI MVDR ABF polynomial
zeros. Ch.~\ref{ch:ucmvdr} introduces the unit circle MVDR ABF
algorithm and compares its performance to the SMI MVDR and diagonal
loaded MVDR ABFs. Ch.~\ref{ch:dzmvdr} introduces the double zero MVDR
ABF algorithm for notch broadening and compares its performance to the
traditional ABFs and the CMT MVDR ABF. Ch.~\ref{ch:conclusion},
summarizes the results of this work and discusses potential questions
the for future investigation.
	

%%% Local Variables:
%%% mode: latex
%%% TeX-master: "main"
%%% End:
